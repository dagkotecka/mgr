\chapter{Title of chapter 2}

Some references \cite{AvrNiemeijer,OpticDiscHaar,HSIRGB,BiologiaVillee}.
\citet{DiabetesKlonoff} has stated that...

The general formatting requirements for the diploma thesis are listed below:
\begin{itemize}
	\item sheet size: A4,
	\item paper orientation: vertical,
	\item font: Arial,
	\item basic font size: 10 pt.,
	\item line spacing: 1.5,
	\item mirror margins:
	\begin{itemize}
		\item top: 2.5 cm,
		\item bottom: 2.5 cm,
		\item internal: 3.5 cm,
		\item external: 2.5 cm,
	\end{itemize}
	\item the thesis text should be justified (aligned to both margins),
	\item each paragraph should begin with a 1.25 cm indentation.
\end{itemize}

The thesis should be prepared for double-sided printing. The page numbering should be in the page footer and centred. The title page should include the author’s (authors’) Statement (Statements) and the page number should not be printed. Page numbering, should be in Arabic numerals using a 9 pt. Arial font. It should begin on page 3 (the Table of Contents) and continue to the last page.

An example of the correct way of presenting information in points (bulleted list) is shown above. Each point (line of text) should be preceded by a bullet. It should begin in the lower case and end with a comma or semi-colon, except for the last point (line of text), which should end with a full stop.

The title of a table should be directly above the table, with a 9 pt. font size and not ended with a full stop, as shown above. Paragraph spacing for the text in the table is as follows:
\begin{itemize}
	\item top 6 pt.,
	\item bottom 0 pt.
\end{itemize}

The correct headings are presented in \hyperref[tab:heading-styles]{Table~\ref*{tab:heading-styles}}.

\begin{table}[h]
	\caption{Sizes and styles of headings}
	\label{tab:heading-styles}
	\begin{tabularx}{\textwidth}{|X|X|X|}
		\hline
		Level of heading	& Example 					& Font size and style \\ \hline
		Heading 1 			& \textbf{1. CHAPTER TITLE}			& 12 pt., CAPITALS, bold \\ \hline
		Heading 2			& \textbf{\textit{1.1. Subchapter title}}		& 10 pt., bold and in italics \\ \hline
		Heading 3			& \textit{1.1.1. Subchapter section}	& 10 pt., italics \\ \hline
	\end{tabularx}
\end{table}

Data should be presented in the table as in \hyperref[tab:heading-styles]{Table~\ref*{tab:heading-styles}}, shown above, i.e. using a 9 pt. font and aligning text to the left edge of the cell.

Table numbering is continuous within the chapter. The table sequence number (table title) is preceded by the word Table and the number of the chapter, ended with a dot (e.g. Table~1.1. Size\ldots). Every table must be referred to in the thesis text, e.g.`\hyperref[tab:heading-styles]{Table~\ref*{tab:heading-styles}}. contains\ldots'.

If a table needs to be continued on more than one page, the table heading should appear on each subsequent page, using the option: TABLE PROPERTIES -> row -> repeat as heading row at the top of every page.

The first paragraph below a table should begin with a top margin of 12 pt.

Lines of text should not end with short prepositions, such as: a, an, the, in, on, etc. In such cases the non-breaking space (NBSP), using ctrl, shift and space, is recommended instead of an ordinary space

The title of a subsequent chapter should always appear on a new page. Every paragraph containing the title of a chapter or subchapter has the following intervals:
\begin{itemize}
	\item top 12 pt.,
	\item bottom 6 pt.
\end{itemize}

Here is some text to get rid of Underfull Badbox. Here is some text to get rid of Underfull Badbox. Here is some text to get rid of Underfull Badbox. Here is some text to get rid of Underfull Badbox. Here is some text to get rid of Underfull Badbox. Here is some text to get rid of Underfull Badbox. Here is some text to get rid of Underfull Badbox. Here is some text to get rid of Underfull Badbox. Here is some text to get rid of Underfull Badbox. Here is some text to get rid of Underfull Badbox. Here is some text to get rid of Underfull Badbox. Here is some text to get rid of Underfull Badbox. Here is some text to get rid of Underfull Badbox. Here is some text to get rid of Underfull Badbox. Here is some text to get rid of Underfull Badbox. Here is some text to get rid of Underfull Badbox. Here is some text to get rid of Underfull Badbox.



\section{Subchapter title}

References in the text to literature should show in square brackets their reference number from the \textbf{Bibliography} or the name of the author/s and year of publication [Kowalski J., 2002] and page number in the case of citations [Kowalski J., 2002, p. 3].

The \textbf{Bibliography} should be arranged alphabetically or in the order items are mentioned in the chapter. If several items of literature are referred to at one point in the text, a dash may be used in the square brackets [1-5] if they are in consecutive order in the \textbf{Bibliography}, or if their order is not consecutive, using commas, as follows: [1, 3, 5].

An example of bibliographic entries is shown in the Bibliography chapter.

Figures are numbered in the order in which they appear in the chapter. In the caption below a figure, the number of the figure is preceded by the abbreviation ‘Fig.’ and the appropriate chapter number, e.g. ‘Fig.~2.1.’

Figures appearing in the thesis should be centred. The spacing in a paragraph containing a figure should be:
\begin{itemize}
	\item top 12 pt,
	\item bottom 0 pt.
\end{itemize}

Every figure must be referred to in the text, for example as follows: `\hyperref[fig:ETI-logo]{Fig.~\ref*{fig:ETI-logo}}. presents\ldots'.

If the thesis is written in English, all the tables and figures must also be presented in the English language.

An example of a properly presented figure and caption is provided below.

The caption below the figure should be centered, have a 9 pt. font size and end with a full stop. The top margin should be 6 pt., the bottom margin 12 pt. and the spacing should be single.

The first paragraph below a figure should have a top margin of 12 pt.



\subsection{Subchapter section heading}

Every paragraph containing a subchapter section heading should have the following spacing:
\begin{itemize}
	\item top 12 pt.,
	\item bottom 6 pt.
\end{itemize}



\section{Subchapter heading}

Variables should be written in italics, e.g. $x, ni, ni+1$, whereas symbols denoting vectors or matrices should be written in bold, e.g. $\mathbf{v}, \mathbf{A}$. The minus sign should directly precede the number without a space, e.g. $-20$. There must always be a space between a numerical value and unit of measurement, e.g. $1\,\mathrm{V}$ or $10\,\mathrm{km}$.

Preferably, the insertion of equations (INSERT -\textgreater{} Equation) directly into the text should be avoided. If there is a possibility of legibly inserting an equation in one line, it should be done as in the following examples: $1/2t^2$ or $e^{2x+1}$. This should be immediately followed by an explanation as to what the particular symbols denote, e.g. where: $t$ --– time [s].

Equations that might become illegible when written in a single line should be written in a separate paragraph, e.g.
\begin{equation}
	\label{eq:exemplary-equation}
	s = v_0 \cdot t + \frac{a \cdot t^2}{2}
\end{equation}
where: \\
\begin{tabularx}{\textwidth}{p{1.25cm}p{0.5cm}X}
	$s$		& ---	& linear displacement with constant acceleration [m], \\
	$v_0$ 	& ---	& initial velocity [m/s], \\
	$t$		& ---	& time of object motion [s], \\
	$a$		& ---	& acceleration [m/s\textsuperscript{2}].
\end{tabularx}

Every equation should be centered. Its number should be preceded by the chapter number and a dot, in parentheses. Every equation has to be referred to in the text, e.g. `Equation~\eqref{eq:exemplary-equation} allows us to estimate\ldots'.

Any footnotes should appear below a line at the bottom of the page,\footnote{Footnotes should be written using Arial 9 pt.} and there numbering should be consecutive throughout the thesis. To make footnotes use NUMBERING OPTIONS -\textgreater{} add footnote.