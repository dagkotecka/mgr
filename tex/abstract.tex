\chapter*{Abstract}

\tab The purpose of this master thesis is to enhance the Scrum methodology using collaborative games. The main focus of this work is on the Retrospective meeting. In this work, we tried to find games that would improve creativity, collaboration between team members and effective communication. Simultaneously, we also focused on the motivation and the opinion of the participants in terms of their understanding of the game. Moreover, we tried to verify whether the methods introduced in the research were more effective than the standard procedures or whether they complement the standard procedures. We attempt to convince the team members that using collaborative games permanently will bring improved results in terms of the project and relationship between team members. The methodology used in this work is Action Research. We implemented the games in Intel Technology Poland to aid them with issues related to creativity, communication and team work. Furthermore, we gathered people to perform the focus group to discover the Scrum methodology in practice. We aimed to obtain what the pros and cons of the Scrum were according to the developers opinion. Apart from that, we were able to find out what Scrum practices they are following, which meetings and roles in Scrum they find useful and what the biggest pitfalls or impediments in the case of the Agile Methodologies are. We discussed the Scrum Methodology with a group of twelve people. During the deployment of the games, we gathered data using survey responses from team members. The implementation of the games was divided into two iterations. The first deployed two games and after a reflection we changed the questions in the survey to extract more valuable characteristics. The second stage implemented five games resulting in a satisfactory outcome.

\tab The majority of the games described have been introduced and we have also retrieved an opinion on the subject. Furthermore, the results were analyzed to achieve a better implementation in the next stage. Due to their value and in order to assure a generic results, some of the games were deployed more than once.

\tab The games have been divided into two categories - fundamental and complementary. The Retrospective meetings are based on the fundamental games and the complementary games are those that support it. In our approach, the Retrospective can only happen if the basic games are used, but it is not possible to perform the meeting based only on the complementary games.

\tab During the realisation of the described research work, thanks to the help of Intel Technology Poland, we were able to create a new game and improve the existing one.

\tab During the collection and analysis of the data, we established that creating a tool will make the Scrum Master's work more efficient. We implemented a web service called Retrospective Analyzer based on the feedback from the surveys. The main functionality of the system was to ease the work of the leaders whilst performing the Scrum Retrospective meeting.

\tab After the project has ended the games implemented in Intel Technology Poland are still being used. This includes not only the teams that participated in the research, but also teams which found out about the our  approach from the team members from the the teams that attended the research. 

\vspace{12pt}
\noindent\textbf{Key words:}

\vspace{6pt}
\noindent collaborative games, Scrum, software engineering, Retrospective, innovation, enhancement