\chapter*{Abstract}

\tab The purpose of this master thesis is to enhance the Scrum methodology using collaborative games. The main element of the research is Retrospective meeting. In the research paper we mainly aimed to find games that would improve the creativity, collaboration between team members and effective communication. Simultaneously, we also focused on the motivation and the opinion of the participants in terms of their understanding of the game. Moreover, our goal was to verify whether the methods introduced in the research work were more effective than the standard procedures or would complement the standard procedures. Our purpose was to convince the team members that using collaborative games permanently brought improved results in terms of the project and relationship between team members. The methodology used in this research paper is Action Research. We implemented the games in Intel Technology Poland to aid them with issues related to creativity, communication and team work. During the deployment of the games we gathered data using surveys from team members. The implementation of the games was divided into two sections. The first deployed two games and after a reflection we changed the questions in the survey to extract more valuable characteristics. The second section implemented five games resulting in a satisfactory outcome.

\tab During collecting the data and analysing it, we established that creating a tool will make the Scrum Master's work more efficient. We implemented a web-service called Retrospective Analyzer based on feedback from the surveys. The main functionality of the system was to ease the work of the leaders whilst performing the Scrum Retrospective meeting.

\tab Furthermore, we gathered people to perform the focus group to discover the Scrum methodology in practice. We aimed to obtain what are the pros and cons of the Scrum in terms of the developers opinion. Moreover, what Scrum practices are they following, which meetings and roles in Scrum they find useful and what are the biggest pitfalls or impediments in the case of the Agile Methodologies. We discussed the Scrum Methodology in a group of twelve people.

\tab This master thesis finishes with a summary of where the contribution has been practices and what can be improved regarding the future of the Scrum.

\vspace{12pt}
\noindent\textbf{Key words:}

\vspace{6pt}
\noindent collaborative games, Scrum, software engineering, Retrospective, innovation, enhancement