\chapter{Summary}
This chapter contains the summary of this master thesis, the contribution that has been made and the future work that might be done.

\section{Contribution}

The goal of this master thesis was to verify whether collaborative games are able to improve the Retrospective meeting. The main enhancement areas were team members involvement, creativity and communication. We also tried simultaneously to increase motivation of team members. Moreover, we asked the participants about their opinion about the game. We required from them answers about the influence of the game comparing to standard procedures, whether they would implement it permanently, does it complement standard procedures and if the game is easy to understand. In the analysis of the retrieved results we tried to respond to the hypothesis raised in the first chapter.

Firstly, we gathered all the data required to start the collaborative games implementation. We found the most interesting ones and deployed them in the Intel Technology Poland site. While searching the games, we had to analyze many retrospective approaches in order to find the most suitable for our case. Two of the approaches presented in this master thesis has been created in collaboration with Intel and its Scrum Master.

Secondly, we deployed the games in two iteration using different sets of questions. The first iteration of the implementation contained two games which were implemented in two separate teams containing 9 and 3 people. The question set has been changed, because after reflection with the supervisor and the study group we decided to improve them in order to retrieve more interesting and less generic characteristics. The second iteration has been implemented in three teams with 9, 3 and 8 participants. Most of the games have been implemented twice, but there were cases when some of them have not been deployed or have been just once, because of internal organisation issues.

After the meeting we asked the participants to fill the survey. We collected and analysed the data and in most of the cases the results of the survey were satisfying and proved the hypothesis that has been raised in the introduction of this master thesis. In just one game, which is very similar to standard procedures, the team members answered that in major fields it does not enhance the Retrospective meeting. Besides that, we can claim with high confidence that the collaborative games increase team members creativity and involvement. Furthermore the participants of the Retrospective meeting were willingly discussing issues in the team and the project while using the collaborative games. What is more, the additional characteristics, in most of the games enhanced. The study showed that using sticky-notes and the blackboard, instead of just writing down the Good Things, Bad Things and Things to Improve, increased the team collaboration and brought more and better results.

Furthermore, the focus group has been performed on the group of twelve people. The conclusions have been a major part in describing Scrum in practise. We gather developers in one room and discussed using earlier prepared set of questions, what they think about this methodology. Team members willingly answered the questions and interesting results have been retrieved.

While collaborating with the Scrum Master in Intel Technology Poland site we decided to create a tool that will base on the study presented in this master thesis. The web-service created for the purpose of this study has been open-sourced and will be expanded in the future. The main functionality, which is game retrieval based on set of questions, has been implemented. The Retrospective Analyzer is already supporting the Scrum Master to find a suitable, for a particular team, game, which is the most important success of this system.

\section{Future work}

This Master Thesis was focused only on Scrum Retrospective and the study about whether the collaborative games enhance the meeting has been fully covered. In the future it would be advised to research other Scrum practices. There are plenty of games for the Planning meeting, Backlog grooming and Daily Stand-ups, where the games could be implemented. It would be interesting to verify whether the collaborative games are able to enhance better estimation of the tasks or prioritization of the user stories. What is more, an intriguing research would be the improvement of Daily Stand-Ups. In this case it would be advised to focus on fitting into the Daily Stand-Up timebox or directing team members to talk only about "what you did yesterday" and "what are you going to do today" instead of digging into implementation details.

Another future work should be expanding the software described in this master thesis. The Retrospective Analyzer would be even more useful tool if it had creating groups functionality, admin panel and allowing multiple team members to affect the resulting game.

\section{Conclusion}

To sum up, in overall the collaborative games enhance Scrum Retrospective meeting and has a positive impact on the team members. The Scrum Master from Intel Technology Poland has been satisfied with the functionality that the complementary software, which has been added to this thesis, offers. The Retrospective Analyzer is being used in the real organization and it fulfills its primary purpose.