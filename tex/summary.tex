\chapter{Summary}
This chapter contains the summary of this master thesis, the contribution that has been made and the future work that might be done.

\section{Contribution}

The goal of this master thesis was to propose and implement a set of collaborative games in order to improve involvement of team member, creativity, communication and team members’ motivation. Moreover, we asked the participants about what their opinion is regarding the game. We investigated what the influence of the game was compared to standard procedures, whether they would implement it permanently, whether it would complement standard procedures and whether or not the game is easy to understand. In the analysis of the retrieved results, we tried to respond to the question "Do collaborative games enhance the Retrospective meeting in terms of the motivation, creativity, team members involvement and communication?".

Firstly, we gathered all the data required to start the collaborative games implementation. We determined which were the most interesting, enhance the problems which occurred in Intel Technology Poland and as a result, deployed them in the Intel site. While searching the games, we had to analyze many retrospective approaches in order to find the most suitable for our case. Two of the approaches presented in this master thesis have been created in collaboration with Intel and its Scrum Master.

Secondly, we deployed the games in two iterations using different sets of questions. The first iteration of the implementation contained two games which were implemented in two separate teams containing 9 and 3 people, respectively. The question set had been changed though, because after reflection with the supervisor and the study group, we decided it needed improvement in order to retrieve more interesting and less generic characteristics. The second iteration has been implemented in three teams with 9, 3 and 8 participants. Most of the games had been implemented twice, but there were cases when some of them had not been deployed or were deployed only once. This was due to internal organisation issues.

Furthermore, the focus group had been performed on a group of twelve people. The conclusions have been a major part in describing Scrum in practice. We gathered developers in one room, and using the aforementioned prepared set of questions, we discussed what they thought about this methodology. The team members willingly answered the questions and as a result, interesting results have been retrieved.

After the Retrospective meeting we asked the participants to fill out a survey. We collected and analysed participants' answers, and in the most cases the results were satisfactory. In just one game, which is very similar to standard procedures, the team members admitted that in major fields, it does not enhance Retrospective meetings. Besides that, we can claim with high confidence that the collaborative games increase both the creativity and involvement of the team members. Furthermore, we found that the participants of the Retrospective meeting were willing to discuss issues both in the team and in the project while playing the collaborative games. What is more, the additional characteristics, in most of the games were enhanced. The study showed that using sticky-notes and the a blackboard increased the team collaboration and brought more and better results, compared to just writing down the Good Things, Bad Things and Things to Improve.

While collaborating with the Scrum Master in Intel Technology Poland in Gdansk, we decided to create a tool that is based on the study presented in this master thesis. The web service created for the purpose of this study has been open-sourced and will be expanded in the future. The main functionality has been implemented, which is game retrieval based on a set of questions. The Retrospective Analyzer is already supporting the Scrum Master to find a suitable game, for a particular team, which we found to be the most important success of this system.

\section{Future work}

This Master Thesis focused only on the Scrum Retrospective and the study about whether the collaborative games enhance the meeting has been fully covered. In the future, it would be useful to research other Scrum practices. There are plenty of games that could be implemented for the Planning meeting, Backlog grooming and Daily Stand-ups. It would be interesting to verify whether the collaborative games are able to enhance better estimation of the tasks or prioritization of the user stories. What is more, an intriguing piece of research research would be the improvement of Daily Stand-Ups. In this case, it would be useful to focus on fitting into the Daily Stand-Up "timebox" or directing team members to talk only about "what you did yesterday" and "what are you going to do today" instead of digging into the implementation details.

Another piece of future work should be to expand the software described in this master thesis. The Retrospective Analyzer would be an even more useful tool if it had a "creating groups" functionality, an admin panel and the possibility of allowing multiple team members to have an effect on the resulting game. 

\section{Conclusion}

To sum up, the collaborative games enhance the Scrum Retrospective meeting and has shown to have a positive impact on the team members. Moreover, after the project has ended the games implemented in Intel Technology Poland are still being used. The teams that have participated in the research shared their knowledge about the proposed approach to the other teams, due to that the teams that did not attend in the research are also using collaborative games. The Scrum Master from Intel Technology Poland has been pleased with the functionality that the complementary software offers, and as a result has been added to this thesis. The Retrospective Analyzer is being used in the real organization and it has been shown to fulfill its primary purpose.