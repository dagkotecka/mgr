\chapter{Introduction}
\section{Context and motivation}
The Scrum is a methodology that can be traced back to 1986, where it was introduced in the article "The New New Product Development Game"\cite{ScrumHarvardBib, ScrumBook}, published by the Harvard Business Review. Back in 1986, companies such as Honda, Canon and Fuji-Xerox were producing world-class results using an all-at-once product development method, which was a scalable, team-based technique and one that emphasized the matter of having teams that are self-organized and empowered. What is more, they outlined the role of management in the development process. The name "Scrum" is not an acronym, it is derived from the sport of rugby, "where it refers to the way of restarting a game after an unintended infringement or in case the ball has gone out of play"\cite{AgileBook}. In the eighties, large companies creating defense and IT projects were failing due to frequency growth, which led to numerous books explaining how to create a better process. The main issue in the projects back in the 1980s was that traditional methodologies were focused mostly on the documentation and producing secondary artifacts. That approach usually resulted in failure of the product, because it did not cover the business requirements. The growth of the companies and technological progress led to creating the Agile Manifesto in 2001\cite{AgileManifesto, AgileBook}. The created principles set contains rules as follows:

"We are uncovering better ways of developing
software by doing it and helping others do it.
Through this work we have come to value:

Individuals and interactions over processes and tools \hfill \break
Working software over comprehensive documentation \hfill \break
Customer collaboration over contract negotiation \hfill \break
Responding to change over following a plan \hfill \break

That is, while there is value in the items on
the right, we value the items on the left more."\cite{AgileManifesto}

The motivation of this master thesis was to verify if collaborative games support Scrum principles\cite{AgileManifesto} such as interactions and collaboration. Moreover, our goal is to confirm whether the collaborative games increase the "continuous attention to technical excellence" and whether it helps "build projects around motivated individuals", who take part in "face-to-face conversations". In this research, we also tried to investigate whether the effectiveness of the team can be improved by the usage of proposed solutions.

According to Morales-Trujillo et al.\cite{MiguelGames}, a game can be seen as an activity, which leads to learning new skills and applying them to overcome challenges, getting rewards or punishments. In this article, serious games are indicated as those which can be played to achieve a precise goal and to provide entertainment. The serious games might be used as a strategy to address different kind of challenges and to avoid obstacles\cite{MiguelGames}.

This thesis' goal is to verify whether it is possible to enhance Scrum Retrospectives using collaborative games.

\section{Problem statement}

The goal of this research is to propose and implement a set of collaborative games to improve communication, motivation, creativity and involvement amongst team members.

\section{Research method}

The research design is an action research method deployed in Intel Technology Poland in Gdańsk in collaboration with the teams working there. The company was experiencing issues in particular fields of Scrum methodology deployment. The main problem that occurred in Intel teams was a lack of involvement during the Retrospective meeting. Other issues were also observed, such as poor creativity, failure to initiate a discussion and the feeling that the meeting was a waste of time that potentially could have been used for the product development. We cooperated with one of the main Scrum Masters in the corporation, Grzegorz Reglinski. Grzegorz is a certified scrum master (Professional Scrum Master), who has over 10 years of experience in working with Agile teams as a developer, product owner and scrum master. He has gained the practical knowledge, while working on Agile projects. His experience was built in Scrum, Kanban and Scrumban teams. From the agile side, Grzegorz is currently managing and supporting over 50 people working together in a project based on micro-services. The practical aim of our action research is to enhance the efficiency of retrospective meetings. Moreover, the research goal is to examine whether collaborative games can positively influence cooperation and motivation of scrum team members. What is more, we aim to verify whether the creativity of developers would increase using the proposed techniques and whether they are more involved, whilst being entertained. The important factor is also the extraction of valuable results using serious games.

To collect the feedback, we used a focus group and a survey.

A focus group is a discussion led by a moderator within a small group of six to ten people. The aim is to generate a maximum number of different opinions and ideas from various people in the time range between 45 to 90 minutes. The focus group is structured around no more than ten carefully predefined questions. We expect that the participants will stimulate and influence each other's thinking and be more open to discussion\cite{FocusGroupBib}.

A survey "is a question or a series of questions in order to gather information about what most people do or think about something"\cite{Webster}. This approach was used to retrieve feedback from team members after collaborative game deployment on the Retrospective meeting.

\section{Related work}

Morales-Trujillo et al. \cite{MiguelGames} wrote a paper on "Improving Software Projects Inception Phase Using Games" and his work was focused on Agile methodologies and their meetings. This master thesis aims to improve only the Scrum Retrospective meeting. Morales-Trujillo's goal was to help developers understand their customers, market and business opportunities. What is more, he aimed for the improvement of everyday issues in the organisation. Our work is strictly focused on human relations enhancement and retrieving valuable outcomes in the case of the project in the Retrospective meeting. 

Another related work is a master thesis written by Mateusz Zakrzewski\cite{Zakrzewski}. He focused on enhancing the requirements elicitation process using collaborative games. The target of the analysis was to improve the creation of the product backlog, sprint planning and sprint review. In our research we focused on the Retrospective meeting, mainly in the case of creativity, team involvement and collaboration between team members. 

There is a strong linkage between Zakrzewski's paper and an article written by David Gelperin \cite{related2} regarding the improvement of the communication and cooperation between the developers and customers, while collecting the requirements. He introduces six serious games which help acquire a deep understanding of user and customer needs, through effective communication and cooperation.

We propose an innovative web service named Retrospective Analyzer, which is able to retrieve a game based on simple questions. The participants of the retrospective meeting are obliged to fill a form and based on their answers the system will suggest the most suitable game. According to our knowledge our system is the first of its kind.

\section{Outline of the thesis}

The thesis contains 7 chapters. The first one describes the aim of the work and is an introduction to the topic. The first section of chapter 2 presents the state of the art. It is focused on how does Scrum looks in theory, the process, what kinds of roles do we distinguish in this methodology and what meetings are required. In the second section of chapter two, the practical Scrum is described. The third chapter contains a description of the fundamental and supplementary games together with their requirements and rules. In the following chapter, games implementation and deployment has been elaborated. What is more, in chapter 4, the results of the team meetings and the outcome from the participants' feedback was presented. In addition, the chapter contains interesting aspects of the games implementation process. Chapter five includes a detailed description of the architecture of the created web service, the way the system works and the evaluation of the Scrum Master. The last chapter summarizes all the work that has been done in this master thesis.
