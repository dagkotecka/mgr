\chapter{Introduction}
\section{Context and motivation}
The Scrum is a methodology can be traced back to 1986, where in article "The New New Product Development Game" \cite{ScrumHarvardBib, ScrumBook}, published by Harvard Business Review. Back in 1986 companies such as Honda, Canon and Fuji-Xerox were using producing word-class results using all-at-once product development method, which was a scalable, team-based technique and emphasized the matter of having teams that are self-organized and empowered, what is more they outlined the role of management in the development process. The name "Scrum" is not an acronym, its name derives from sport of rugby, where the way of restarting a game after an unintended infringement or in case the ball has gone out of play. In the eighties, large companies creating defense and IT projects were failing due to growing frequency, which headed to numerous books how to create a better process. The growth of the companies and technological progress led to creating in 2001 agile manifesto \cite{AgileManifesto, AgileBook}. The agile manifesto contains rules as follows:

"We are uncovering better ways of developing
software by doing it and helping others do it.
Through this work we have come to value:

Individuals and interactions over processes and tools \hfill \break
Working software over comprehensive documentation \hfill \break
Customer collaboration over contract negotiation \hfill \break
Responding to change over following a plan \hfill \break

That is, while there is value in the items on
the right, we value the items on the left more." \cite{AgileManifesto}

A game can be seen, accordingly to Miguel Ehécatl Morales-Trujillo in his article "Improving Software Projects Inception Phase Using Games", as an activity which leads to learning new skills and applying them to overcome challenges, getting rewards or punishments. In the article serious games are distinguished as those which can entertain, even though it is not their primary purpose, and may be played seriously. The serious games might be used as a strategy to address different kind of challenges and sorting obstacles \cite{MiguelGames}.

In this thesis, the main purpose of the serious games is to verify, whether it is possible to enhance the Scrum using collaborative games.

\section{Problem statement}

The goal of this research work is to propose and implement a set of collaborative games for improving communication, motivation, creativity, involvement and introduce a innovate approach in Scrum teams.

\section{Research method}

The research design is an action research method deployed in Intel Technology Poland site in Gdańsk in collaboration with its teams. The company was experiencing issues in particular fields using Scrum methodology. We cooperated with one the main Scrum Masters in the corporation, Grzegorze Reglinski. Grzegorz is a certified scrum master (Professional Scrum Master), has over 10 years of experience of working in Agile environment as a developer, product owner and scrum master. He gained practical knowledge, while working on agile projects and solving problems. His experience was built in Scrum, Kanban and Scrumban teams. Grzegorz currently manages and supports, from the agile side, over 50 people working together in a project based on micro-services. The practical aim of action research is to enhance performing retrospective meetings. Moreover, the research goal is to examine whether the games are positively influencing cooperation and motivation of the participants. What is more we aim to verify if creativity of team members was increased using the proposed techniques and if their are more involved, while being both entertain and while extract valuable results using serious games.

\section{Related work}

Miguel Ehécatl Morales-Trujillo wrote a paper on "Improving Software Projects Inception Phase Using Games" and his work was focused on all the Scrum meetings, this master aims to improve only the retrospective meeting. Miguels goal was to help understand customers, market, business opportunities and to improve everyday issues in the organisation. This work is strictly focused on human relations enhancement and retrieving valuable outcome in case of the project in the retrospective meeting.

The Synergy-Analyzer introduced is a innovate web service which is able to retrieve a game based on simple question. The participants of the retrospective meeting are obliged to fill a form and basing on their answers the system will retrieve a best, suitable game for their issues in the team. The solution has no related work, this a first kind of system.

\section{Outline of the thesis}

The work contains 7 chapters, the first one describes the aim of the work and is an introduction to the topic. The next chapter contains knowledge from the literature, books, articles and publications. It is focused on how does scrum looks in theory, the process, what kinds of roles do we distinguish in this methodology and what meetings are required. In the second section of chapter two the practical scrum has been described. The third chapter contains description of fundamental and supplementary games, what are the requirements and rules. In the following chapter games deployment has been elaborated, the results of the teams meeting, the outcome from the participants' feedback and what was observed while implementing the game. Moreover the chapter has been divided into two sections, using two different sets of questions for feedback retrieval. The chapter five includes web service detailed description of the architecture and the way it works. The next chapter contains collected data about how the system was evaluated and the outcome of the deployment. The last chapter summarizes all the work that has been done in this master thesis.














