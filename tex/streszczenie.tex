\chapter*{Streszczenie}

\tab Celem niniejszej pracy magisterskiej jest usprawnienie metodyki Scrum za pomocą gier zespołowych. Głównym elementem badań jest spotkanie Retrospektywa. W tej pracy badawaczej szczególnie skupiliśmy się na wyszukiwaniu gier, które usprawniłyby kreatywność, współpracę między członkami zespołu oraz efektywną komunikację. Równolegle, skoncentrowaliśmy się na motywacji oraz opinii uczestników w kwestii ich zrozumienia gier. Co więcej, naszym celem była weryfikacja czy wprowadzone w tej pracy badawczej gry były bardziej efektywne niż standardowe procedury i czy mogłyby stanowić uzupełnienie standardowych procedur. Postawiliśmy sobie za zadanie, aby przekonać członków zespołu, że wprowadzenie gier zespołowych na stałe poprawi relacje międzyludzkie w zespole, jak również przyniesie korzyści dla projektu. Metoda badawcza użyta w tej pracy magisterskiej to Action Research. Wprowadziliśmy gry zespołowe w firmie Intel Technology Poland, aby pomoc im poprawić kreatywność, komunikację oraz współpracę w zespołach deweloperskich. Podczas wprowadzania gier zbieraliśmy dane z ankiet od członków zespołu. Implementacja gier była podzielona na dwie fazy. Pierwsza faza wprowadziła dwie gry i po refleksji z zespołem badawczym i promotorem zmieniliśmy pytania w niej w celu uzyskania bardziej wartościowych cech. Druga iteracja implementowała pięć gier, które w rezultacie przyniosły zadowalające wyniki.

\tab Większość opisanych w tej pracy gier zostało wprowadzonych oraz otrzymaliśmy opinie na ich temat. Następnie wyniki przeanalizowano w celu jak najlepszej implementacji w kolejnych fazach. Niektóre gry z uwagi na ich wartościowość były wprowadzane więcej niż raz, co skutkowało tym, że nie wszystkie opisane gry zostały zaimplementowane w praktyce.

\tab Gry zostały podzielone na dwa rodzaje - podstawowe oraz uzupełniające. Gry podstawowe to takie na których opiera się całe spotkanie Retrospektywa, natomiast uzupełniające to takie, które je wspierają. Retrospektywa może odbyć się tylko i wyłącznie używając gier podstawowych, lecz nie ma możliwości przeprowadzania jej bazując tylko na uzupełniających.

\tab W trakcie realizacji opisywanej pracy badawczej, dzięki współpracy z firmą Intel Technology Poland, udało nam się stworzyć jedną zupełnie nową grę oraz usprawnić istniejąca.

\tab Podczas zbierania danych i analizowania ich, postanowiliśmy stworzyć narzędzie, które będzie wspierać i pomoże efektywniej pracować Scrum Masterowi. Zaimplementowaliśmy serwis internetowy, Retrospective Analyzer, bazując na informacjach zwrotnych wyciągniętych z ankiet. Główną funkcjonalnością systemu jest ułatwienie pracy liderów, podczas przeprowadzania spotkania Scrum Retrospective.

\tab Ponadto zebraliśmy ludzi i odbyliśmy spotkanie używając formy badawczej focus group w celu wydobycia informacji na temat Scruma w praktyce. Dzięki temu podejściu dowiedzieliśmy się jakie są wady i zalety Scruma w oczach członków zespołu. Oprócz tego udało nam się dowiedzieć za jakimi Scrumowymi praktykami podażają, jakie role i spotkania uważają za potrzebne oraz jakie są największe ich potknięcia związane ze metodykami zwinnymi. Dyskusja na temat metodyki Scrum odbyła się w grupie dwunastu osób.

\tab Praca zawiera również dokładny opis Scruma w teorii, jakie są jego zasady oraz praktyki. Ponadto zamieszczone zostały również dokładne opisy procesów, ról oraz spotkań, które są podstawą metodyki Scrum.

\tab Owa praca kończy się podsumowaniem w której zawarty jest nasz wkład w tą pracę magisterską oraz jakie usprawnienia są możliwe w przyszłosci oraz powinny być wykonane z udziałem metodyki Scrum. 



