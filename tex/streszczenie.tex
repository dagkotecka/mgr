\chapter*{Streszczenie}

\tab Celem niniejszej pracy magisterskiej jest usprawnienie metodyki Scrum za pomocą gier zespołowych. Głównym elementem badań jest spotkanie Retrospektywa. W tej pracy badawaczej szczególnie skupiliśmy się na wyszukiwaniu gier, które usprawniłyby kreatywność, współpracę między członkami zespołu oraz efektywną komunikację. Równolegle, skoncentrowaliśmy się na motywacji oraz opinii uczestników w kwestii ich zrozumienia gier. Co więcej, naszym celem była weryfikacja czy wprowadzone w tej pracy badawczej gry były bardziej efektywne niż standardowe procedury i czy mogłyby stanowić uzupełnienie standardowych procedur. Postawiliśmy sobie za zadanie, aby przekonać członków zespołu, że wprowadzenie gier zespołowych na stałe poprawi relacje międzyludzkie w zespole, jak również przyniesie korzyści dla projektu. Metoda badawcza użyta w tej pracy magisterskiej to Action Research. Wprowadziliśmy gry zespołowe w firmie Intel Technology Poland, aby pomoc im poprawić kreatywność, komunikację oraz współpracę w zespołach deweloperskich. Ponadto zebraliśmy ludzi i odbyliśmy spotkanie używając techniki focus group w celu wydobycia informacji na temat Scruma w praktyce. Dzięki temu podejściu dowiedzieliśmy się jakie są wady i zalety Scruma z perspektywy praktyków. Oprócz tego udało nam się dowiedzieć za jakimi Scrumowymi praktykami podażają, jakie role i spotkania uważają za potrzebne oraz jakie są największe ich potknięcia związane ze metodykami zwinnymi. Dyskusja na temat metodyki Scrum odbyła się w grupie dwunastu osób. Podczas wprowadzania gier zbieraliśmy dane z ankiet od członków zespołu. Implementacja gier była podzielona na dwie fazy. Pierwsza faza wprowadziła dwie gry i po refleksji z zespołem badawczym i promotorem zmieniliśmy pytania w niej w celu uzyskania bardziej pogłębionej analizy. Druga iteracja implementowała pięć gier, które w rezultacie przyniosły zadowalające wyniki.

\tab W trakcie realizacji niniejszej pracy magisterskiej wprowadziliśmy większość opisanych gier oraz została przeprowadzona ankieta na ich temat. Następnie wyniki zostały przeanalizowane w celu ich jak najlepszej implementacji w kolejnych iteracjach. Niektóre gry głównie w celu zapewnienia większej generyczności wniosków oraz z uwagi na ich wartościowość, były wprowadzane więcej niż raz.

\tab Gry zostały podzielone na dwa rodzaje - podstawowe oraz uzupełniające. Gry podstawowe to takie na których opiera się całe spotkanie Retrospektywa, natomiast uzupełniające to takie, które je wspierają. Retrospektywa może odbyć się używając wyłącznie gier podstawowych, lecz nie ma możliwości przeprowadzania jej bazując tylko na uzupełniających.

\tab W trakcie realizacji opisywanej pracy badawczej, dzięki współpracy z firmą Intel Technology Poland, udało nam się stworzyć jedną zupełnie nową grę oraz usprawnić istniejącą.

\tab Podczas zbierania danych i analizowania ich, postanowiliśmy stworzyć narzędzie, które będzie wspierać i pomoże efektywniej pracować Scrum Masterowi. Zaimplementowaliśmy serwis internetowy, Retrospective Analyzer, bazując na informacjach zwrotnych wyciągniętych z ankiet. Główną funkcjonalnością systemu jest ułatwienie pracy liderów, podczas przeprowadzania spotkania Scrum Retrospective.

\tab Po zakończeniu projektu gry zaimplementowane w firmie Intel Technology Poland są  dalej wykorzystywane. Dotyczy to nie tylko zespołów w których badania zostały przeprowadzone, ale również zespołów, które dowiedziały się o proponowanym przez nas podejściu od członków zespołów badanych.

\vspace{12pt}
\noindent\textbf{Słowa kluczowe:}

\vspace{6pt}
\noindent gry zespołowe, Scrum, inżynieria oprogramowania, Retrospektywa, innowacja, usprawnienie

